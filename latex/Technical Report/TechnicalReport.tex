\documentclass[11pt,a4paper,oneside]{article}
\usepackage{fullpage}
\usepackage{hyperref}
\usepackage{amsmath}
\usepackage{comment}
\usepackage[pdftex]{graphicx}
\usepackage{url}
\usepackage[all]{xy}
\usepackage{caption}
\usepackage{subcaption}
\usepackage{chngpage}

\begin{document}
\title{\textbf{Dutch Nao Team}}
\author{\textbf{Technical Report 2011}\\ 
\url{http://www.dutchnaoteam.nl}\\\\
Faculty of Science, Universiteit van Amsterdam, The Netherlands}
\date{\today}
\maketitle

\begin{figure}[htb]    	
\centering
\fbox{\includegraphics[width=1\columnwidth]{TeamLarge.jpg}}
%\caption{The Dutch Nao Team}
\label{fig:TeamPhoto}
\end{figure}

\begin{figure}[htb]    	
\centering
\includegraphics[width=0.3\columnwidth]{DNTLogo.png} % logo may look better when smaller
%\includegraphics{DNTLogo.png} % real size of image
%\caption{The Dutch Nao Team Logo} % looks better without caption
\label{fig:DNTLogo}
\end{figure}

\newpage
\tableofcontents
\newpage
\section{Introduction}
The Dutch Nao Team consists of Artificial Intelligence students from the Universiteit van Amsterdam, supported by a senior staff-member. The Dutch Nao Team is the continuation of the Dutch Aibo Team; a cooperation of several Dutch universities who were active in the predecessor of the Standard Platform League \cite{AIBOTeam2006, AIBOTeam2005, AIBOTeam2004}. The Dutch Aibo Team has been successful, both in the competition and with a number of publications \cite{ArnoudJurgen2009, RoboticDog2008, Panoramic2007, ColorLearning2006, Localization2006, Perception2005}.
The Dutch Aibo Team always published their source-code online including a technical report about the innovations \cite{Visser2006}.
The Dutch Nao Team will continue this tradition. The Dutch Nao Team debuted in the SPL competition at the German Open 2010 \cite{DutchNaoTeamTDP2010}.\\

Since the team consists of twenty-three members, the responsibilities have been distributed over our team as follows:
\begin{description}
\item[Supervisor:] Dr. Arnoud Visser % (assistant professor Intelligent Systems Lab Amsterdam) %assistent professor is a combination of researcher and teacher
\item[Coordinator:] Duncan ten Velthuis
\item[Vice-coordinators:] Camiel Verschoor, Auke Wiggers
\item[Programmers:] Boudewijn Bod\'ewes, Michael Cabot, Erik van Egmond, Eszter Fodor, Sharon Gieske, Robert Iepsma, Sander Jetten, Osewa Jozefzoon, Anna Keune, Elise Koster, Hessel van der Molen, Sander Nugteren, Tim van Rossum, Richard Rozenboom, Justin van Zanten, Maurits van Bellen, Steven Laan
\item[Public Relations:] Merel de Groot, Timothy Dingeman
\item[Sponsors:] Romy Moerbeek
\end{description}

\section{Qualification}
The team that participated in the German Open 2010 \cite{DutchNaoTeamTDP2010} consisted of only two students. Over the summer the team was extended to twenty-three students. This team had their first meeting in October 2010. 

That day the commitment was made to travel to Istanbul for the RoboCup 2011. In December a qualification document \cite{DutchNaoTeamQualification} and qualification video\footnote{Available at \url{http://www.youtube.com/watch?v=p9rWsUJ1lZ4}} were produced. The qualification document described the research that was to be performed in the upcoming months and how this was to be achieved.

\section{Competitions}

\subsection{Rome Open}

\begin{figure}[htb]    	
\centering
\fbox{\includegraphics[width=0.5\columnwidth]{TeamPhotoRome.jpg}}
\caption{The Dutch Nao Team in Rome}
\label{fig:TeamPhotoRome}
\end{figure}

The teams that competed in the Rome Open Standard Platform League were:
\begin{enumerate}
\item SPQR+UChile (Italy + Chile)
\item SpelBots (USA)
\item Austrian Kangaroos (Austria)
\item Portuguese Team (Portugal)
\item Dutch NAO Team (The Netherlands)
\item Los Hidalgos (Spain)
\end{enumerate}

\subsubsection{Matches}
The first match was lost against the Austrian Kangaroos (0-1), and the second match ended in a tie against Los Hidalgos (0-0). The third match was again against the Austrian Kangaroos (0-3). The last match was again a tie against Los Hidalgos (0-0).
Later the Spanisch and Austrian team played the final of the Rome Open, so it was a pitty that the Dutch Nao Team played no matches against teams in the other pool. 
The team gained experience from the matches and due to this the implementation of the program was improved.

\subsubsection{Developments}
Due to a local power cut the game controller could not broadcast any game information as all servers were down. Therefore all teams were forced to use a button interface. The team was not prepared as the button interface was not yet implemented. An attempt was made to make one overnight, but this was unsuccessful.

\subsection{Iran Open}
\begin{figure}[htb]    	
\centering
\fbox{\includegraphics[width=0.5\columnwidth]{TeamPhotoTeheran.jpg}}
\caption{The Dutch Nao Team in Teheran}
\label{fig:TeamPhotoTeheran}
\end{figure}

A total of five teams competed in the Iran Open Standard Platform League:
\begin{enumerate}
\item Nao Team Humboldt (Germany), first place
\item Cerberus (Turkey), second place
\item MRL (Iran), third place
\item Kouretes+Noxious (Greece/United Kingdom, Oxford joined to form this team)
\item Dutch Nao Team (our team)
\end{enumerate}

\subsubsection{Matches}
Due to bad preparation and an unfinished program loader, which is required to start up the program automatically, most of the matches resulted in failure with either no robots on the field or robots that were initiated manually.
 Two matches were last, one match ended in a tie. These matches were against Cerberus (0-4), NTH (0-2) and Kouretes-Noxious (0-0; lost through penalty shootout). Although not visible in the results, many improvements were added to the code.
Again, this was a good opportunity to experience real matches and get used to the process of a tournament. During this tournament the team came across some essential missing elements in the program.

\subsubsection{Developments}
The most important development was that the chin camera was used instead of the camera on the forehead. It was quite a surprise when the team found that the Nao can see the ball when standing directly in front of it, while the team (and even the scientific advisor) had always assumed that the forehead camera was the only one in use.

Significant progress was made throughout the week, especially in the main program controlling the flow of states and the vision department (goal recognition). At the start of the tournament the Aldebaran's balltracking module was used. However, a different method was implemented, which used Python OpenCV (an image processing library).

\subsection{RoboCup Istanbul}
\begin{figure}[htb]    	
\centering
\fbox{\includegraphics[width=0.5\columnwidth]{TeamPhotoIstanbul.jpg}}
\caption{The Dutch Nao Team in Istanbul}
\label{fig:TeamPhotoIstanbul}
\end{figure}

Twenty-seven teams from all over the world competed in the RoboCup 2011. We've played against a few of them:
\begin{enumerate}
\item Austen-Villa (Texas), lost, 0-5
\item Nao Team Humboldt (Germany), one draw, 0-0 and one loss, 0-3
\item SPQR+Chile (Italy and Chile), match ended 1-1, won this match through penalty shootout. This was by far the most exciting and the most visually unattractive match.
\item B-Human (Germany), lost, 0-10
\item UPennalizers (Pennsylvania), lost, 1-3 (note: one own goal)
\item MRL, lost practice match, 0-3
\end{enumerate}
From this opponents three teams reached the quarter finals, and one was the later champion.
Each of these matches posed a chance to see the programs capability of coping with real life conditions (instead of testing in ideal situations). The only matches in which our robot team suffered from severe technical problems were the matches against SPQR+Chile and B-Human. Tweaking and bug fixing was done in between matches while larger changes were made during the night following the matches of that day. It should be noted that the match against B-Human was played with code that was not yet tested in a practice match.

\subsubsection{Developments}
Major progress was made in the main program, goalrecognition, the gamecontroller-listener and buttoninterface, apart from tweaking and bugfixes.

The so-called Soul, where state and phase changes are specified, is the Dutch Nao Team's approach to the game problem. It can be seen as a state machine where situations in the game are represented as the states (see section \ref{sec:architecture}). Several improvements were made, apart from tweaking and bugfixes:
\begin{itemize}
\item Better use of balltracking throughout the program (e.g. more checks to see if ball is still there or keeping track of ball at all time)
\item Initialization possible in Set state instead of initializing variables in Initial. The former resulted in technical difficulties during a penalty shootout.
\item Keeper uses six instead of eight ball-locations, resulting in faster but slightly less accurate keeping. Possible because of more precise balltracking.
\item Stricter conditions for Nao to kick a ball, to make sure the Nao does not kick without a ball being there. 
\end{itemize}

The goalrecognition algorithm was rewritten entirely, resulting in faster goalrecognition (see section \ref{sec:goalrecognition}). This improved the algorithm with a factor ten as the algorithm is more efficient.

The gamecontroller was functional before Istanbul but caused problems due the presence of several other gamecontrollers. Because of their presence the robots started listening to the other gamecontroller causing the program to crash. This phenomenon was solved by reading out the incoming message of the gamecontroller and check whether it broadcasted the team id.

The button interface was also functional before Istanbul but caused minor problems due to variation in the pressing of the chest button. Some referees pressed the button longer than others causing the robots to switch off entirely or play on without being penalized. This problem has been partially solved by instead of checking if a button is pressed, check if it has been released.

\subsubsection{Open Challenge}
\label{sec:zoneblocking}

For the Open Challenge the ZoneBlocking project was presented. This section describes what it is and how it is performed (see Appendix~\ref{app:open_challenge_june}).\\\\
{\bf Relative positioning using trigonometry}\\
To position itself in the desired spot, a Nao needs to not only recognize their current position, but also the desired spot to move to. This was achieved by using copious amounts of trigonometrical techniques.\\\\
{\bf Movement and positioning}\\
After scanning for environmental features, calculating the Nao's relative position to them, and doing the math involved in finding the desired standing location, certain motions were necessary to get there. Since the Nao's walking functions curve slightly to the left and have some overshoot, it was decided that it would be best if small movements were made of about twenty centimetres each. For each twenty centimetres walked, the Nao will stop, turn facing the goal, and re-estimate its walking path. Then, it will continue on its way considering the new measurements. During one of these checks, if the Nao is found standing within the error margin of twenty centimetres, the walk cycle terminates. The Nao will then determine its angle to the goal and use this information to turn away from it, facing the opponents' goal.

This, in contrast to the initial positioning, is done in one move, to block the line between the goal and the opponent. Using the relative location of the attacking Nao and the assumptions to the location of the (unseen) goal, it is possible to move the Nao into the opponent's line while wasting as little time as possible. This gives a set of blobs that should correspond to opposing Nao’s waistbands. These blobs are used to extract the locations of the left and right edges, using these to calculate the angles to the defending Nao (the Dutch Nao Team had already done calculations on this, so for this step the pre-existing method was simply used). The width of the waistband will also directly correspond to the distance of the Nao. 

During the competition the code of ZoneBlocking was adjusted to the circumstances in Istanbul. Unfortunately, it was too difficult to get the relative position to the goal stable enough for a demonstration during the Open Challenge. As alternative, an demonstration based on Dynamic Tree Localization was proposed (see Appendix~\ref{app:open_challenge_july}). The proposal got rejected and a presentation about ZoneBlocking was given without a demonstration.

\subsubsection{Research Presentation}
Because only a limited amount of time (3 minutes) was given to present about a current research topic, a presentation was given like an Ignite talk\footnote{Scott Berkun - "Why and How to Give an Ignite Talk", Ep 19. Video uploaded 23 jun 2009 \url{http://www.youtube.com/watch?v=rRa1IPkBFbg}}. 
The research topic was a short summary of the contribution of the team to the RoboCup Iran Open Symposium 2011 \cite{Visser2011rios}, as indicated in Fig.~\ref{fig:TitlePresentation}.

\begin{figure}[htb]
\centering
\fbox{\includegraphics[width=0.5\columnwidth]{ComparisonSLAMrobocup2011TitlePage.png}}
\caption{Title page of the Research Presentation}
\label{fig:TitlePresentation}
\end{figure}

In the presentation was shown that a RoboCup soccer field is a popular benchmark for localization, but is only sparsely used as a benchmark for SLAM-algorithms. 
With a limited size this benchmark has no scaling issues, while the uncertainty in the sensor and motion model are high enough to be challenging, which was the message to take home as indicated in Fig.~\ref{fig:ConclusionPresentation}. With the same sensor and  motion model three SLAM algorithms were compared: EKF-SLAM, FastSLAM 1.0 and FastSLAM 2.0 \cite{Thrun2005}. In this small world the classic EKF-SLAM algorithm gave the best result, as expected.

\begin{figure}[htb]    	
\centering
\fbox{\includegraphics[width=0.5\columnwidth]{ComparisonSLAMrobocup2011Conclusion.png}}
\caption{Conclusion of the Research Presentation}
\label{fig:ConclusionPresentation}
\end{figure}
\newpage
\section{Program}
The program used in the RoboCup 2011 is written solely in Python. It is compatible with Python versions 2.61. OpenCV for Python is used, mainly for image processing. The files used are all original files written by members of the Dutch Nao Team. Different files are used for different sections, for instance, motions are all specified in a single file and not in the file containing the main program.
\subsection{Architecture}
\label{sec:architecture}
The program solves the game problem using a double state machine in the file \texttt{soul.py}. At the highest level are the \textit{states}, which are the game states as specified by the gamecontroller. 
At a lower level are the \textit{phases}, which are specific situations during a game. Transitions between the states and phases are made based on external input (conditions). 
For state-transtitions, the external input is a message from the GameController and/or the button interface (see section \ref{sec:communication}). 
For phase-transitions, the external input is based on observations (see section \ref{sec:vision}). 

%\begin{displaymath}
%    \xymatrix{								&	& \texttt{BallFound} \ar@/^/[ddll]^{\texttt{ball lost}}\ar@/^/[dddd]^{\texttt{ball close}} \\
%									&	&			\\
%	  \texttt{BallNotFound} \ar@/^/[uurr]^{\texttt{ball seen}}	&	&			\\
%              								&	&			\\
%									&	&\texttt{Kick}\ar@/^/[uull]^{\texttt{ball lost}}\ar@/^/[uuuu]^{\texttt{ball too far}}  			}
%\end{displaymath}

\begin{figure}
\centering
\includegraphics[width=12cm]{Conceptmaps.jpg}
\caption{The phases for both player (top) and keeper (bottom) and conditions for transition between phases.}
\label{fig:phases}
\end{figure}

In each of these phases, checks are made to see if conditions for transition are met (e.g. if in \texttt{BallFound} phase and ball close enough for a kick, go to \texttt{Kick} phase). An overview of these phases and conditions can be found in figure \ref{fig:phases}. An advantage of this approach is that the main problem, playing a game of soccer, is divided into smaller problems, which can be solved independently and locally. A second advantage is that phases can be tested without executing the entire program. In effect, the tests are less time-consuming. On the other hand, the Nao is not able to process circumstances that are not specified in a phase. The solution implemented is a straightforward one where phases can not be interrupted until they have finished. This would not, or less so, be the case if a single program covered the entire game problem. A short explanation of every phase in our program:\\\\
\textbf{BallFound} \\
Contains a findBall call to vision, which returns an x- and y-value representing the relative location of the ball (if found). Uses the found location to move Nao towards the ball. Transition conditions: Ball not found or ball close (close being close enough for a kick). How close this is depends on circumstances of the field (lighting changes interpretation of distance, how fast the Nao walks depends on the carpet, etc.) \\\\
\textbf{BallFoundKeep} \\
Similar to \texttt{BallFound}. The main differences are that the keeper does not walk towards the ball and that old positions of the ball are tracked as well. It is possible to derive speed and direction of the ball using these old positions and some geometry. If two or more positions are known, a function through these points $f(y)$ will intersect with the Nao's $y$-axis in $(y,0)$. The direction of the ball, which is the direction in which the Nao should dive, is determined by that specific $y$. An estimate of the speed at which the ball is moving towards the Nao ($x$-direction) is derived by taking two ballpositions and subtracting the $x$-positions. Transition conditions: Ball moving towards keeper (speed high enough) or ball close (low speed).\\\\
\textbf{BallNotFound} \\
When a player can not see the ball, the head moves around to scan for it. If, after a full scan, the ball is still not found, Nao turns around. Transition condition: Ball found.\\\\
\textbf{BallNotFoundKeep}\\
Similar to \texttt{BallNotFound}, the difference is that the keeper does not turn when a ball is not found. Transition condition: Ball found.\\\\
\textbf{Kick}\\
Player-only phase. Player is close to the ball and scans for goal. Player takes action if a goal is found, action depends on which goal is found. Angle towards goal is used as input for the kicking motion. Transition conditions: Ball lost or phase finished (ball kicked). A special transition was added later: If the ball somehow moves during the kick phase and is too far to kick, move to the BallFound phase.\\\\
\textbf{BallApproaching}\\
Keeper-only phase. Ball moves at high speed towards the keeper in a certain direction \texttt{dir}. Keeper takes action depending on \texttt{dir} (diving or sidestepping, left or right). Transition condition: Phase finished. \\\\
\textbf{InGoalArea}\\
Keeper-only phase. When the ball is in the goalarea, the keeper walks towards it, kicks it and walks back. As returning to the center of the goal is still impossible for us, this phase is not used in the current program.\\\\
\textbf{Unpenalized}\\
When a player or keeper is unpenalized, it moves to this phase first. It tells the player to walk 2 meters in x-direction (forward). Player/keeper then goes to phase ReturnField.\\\\
\textbf{ReturnField}\\
This phase can only be reached from the \texttt{Unpenalized} phase. While the player is walking, search the ball. Transition conditions: Ball found or end of walk.\\

Keepers behave differently from other players in the field. Therefore, a similar approach is used but with different names for the phases (e.g. \texttt{BallFoundKeep} instead of \texttt{BallFound}). This shows another disadvantage of our approach: When a bug is found or a change is made in the phases for regular players, it has to be changed twice, once for the normal phases and once for the keeper phases. However, different behaviour for the player and keeper can be made, where this is needed (e.g. emphasize movement along y-axis for keeper to protect the correct side of the goal, whereas a player will emphasize movement along the x-axis to reach the ball faster). A phase contains the following: 
\begin{itemize}
\item A call to receive visual information and, if needed, further processing of this information
\item One or more condition checks for transition to other phases
\item One or more motion calls
\item Optional: print statements for debugging purposes
\end{itemize}
The exception to this is the \texttt{Unpenalized} phase. This phase cannot be reached from any other phase when in the \texttt{Playing} state and it does not ask for visual input. It is reached when a player or keeper was in the \texttt{Penalized} state and is unpenalized. Its sole purpose is to make this player then walk two metres forward (in the direction it is facing, this should be towards the field) and thus prevent it from staying at the sideline. 

As each phase is implemented as a Python function, it can be called independently. The system thus works as a loop calling the function corresponding to a game phase. It should be noted that the phase can be changed in any state but the corresponding function is only called in the Playing state. Also, a keeper that is penalized will reach the \texttt{Unpenalized} and \texttt{ReturnField} phase successively, meaning it turns into a regular player after being penalized.  

Calls to other aspects of the program (vision and motion) are made through interfaces. These interfaces are specified in different files. DNT chose to use this approach instead of regular motion calls to make the main program (\texttt{soul}) easier to understand. An example: Function \texttt{StandUp()} in \texttt{motionInterface.py} checks the pose of the Nao and then takes action corresponding to this pose. Instead of checking the pose in the main program, this entire process is specified in a different file and can be initiated with a simple function call. 

\subsection{Vision}
\label{sec:vision}
% Michael
The ball- and goal-detection are based on color filters. Prior to a match calibration is needed to set the HSV-color-ranges of the red ball, yellow goal and blue goal. These ranges are then used to filter the image taken by the Nao resulting in a grayscale image. When detecting the ball the filtered image is smoothed to remove any noise. If the brightest pixel passes a threshold then this is assumed to be the ball. The position of the ball is calculated by projecting that pixel onto the ground using the location and orientation of the camera when the snapshot was taken. The goal is detected by appending vertical lines in the filtered image to form blobs. These blobs are tested to see if any describes a goalpost. If two posts are found then the angle towards the goal is calculated.

All the vision components were set up so to run parallel with each other. This was removed in the latest release of the program to prevent overheating of the Nao's head.  

\subsubsection{Blob and Nao detection}
In order to recognize features of the environment, a scan is done for blobs of colour in the snapshots the Naos make while walking. To do this, first each picture is corrected for varying lighting conditions. This is done by taking a sample snapshot and finding the pixel that most approximates white. Luckily, since the robots are playing a game of soccer, there will almost always be a white field line somewhere in the snapshot for sampling. The white point is then assumed to be white and all other colors are corrected to make this pure white. This will give consistently coloured snapshots to work with.

Then, the pictures are scanned for colours in a certain range (for example, yellowish colours for the yellow goal). These color ranges are pre-set according to tests, due to correcting these colours, the pictures are consistent enough to make this assumption. A new black and white picture is made representing this scan, with white pixels denoting a positive pixel and black denoting a negative. This gives a black image with the shape of the object that is scanned for in white, and usually some noise. Then each line of the image is checked for lines of one or more white pixels. These pixels represent a blob on a certain line, line blobs. The data on all of these line blobs is stored along with a unique ID value. Due to this line blobs can be easily re used.

Once the entire image is scanned, this will result in a database filled with line blobs. The next step is to convert these line blobs into two-dimensional blobs. This is done by looping over all the line blobs and checking if any of the line blobs, on a line below the current line, are within the horizontal range of the current line blob. If this is the case these line blobs are mapped to one ID value. Once this is done, all the blobs with the same ID are found and the new maximum edges are calculated. This is stored as a new blob with a new ID, along with this data.

To detect a Nao, all insignificantly sized blobs are discarded; this filters most of the less significant noise. Sometimes, faraway Naos are also discarded during this step; however, this is not a real problem since opponents this far from our goal are no immediate threat. This does leave the bigger noise groups in the picture. To determine which of these objects are actual Naos, a check is done on three lines above and two below each blob. If the detected blob does indeed belong to a Nao, each of these lines should be (approximately) white.

\subsubsection{Goal Recognition Algorithm}
\label{sec:goalrecognition}
The following algorithm is based on the shape of the goal. The goal is defined by its two poles. The poles in the image are rectangular with a length and a width defined by thresholds. Goal recognition is performed on a color filtered image. The color can be either blue or yellow. In short, to check if the poles are present, the algorithm searches the image in vertical direction for lines long enough to be poles. Then it checks if the lines are adjacent and the width is wide enough to be a goal.\\\\ 
Algorithm step-by-step (the threshold $t$ is 20)
\begin{itemize}
\item Filter the image based on color.
	\begin{itemize}
	\item Form a dict of vertical lines.
	\item Iterate from top down steps of ten ($0.5t$) pixels, start at ten.
	\item When a true pixel is found go back nine ($0.5t-1$) pixels (remember the step of ten).
	\item Start iterating by one pixel until a true pixel is found.
	\item Then count until twenty ($t$) pixels are found in a row or until a false pixel is found.
	\item When a false pixel is found retry at the last known pixel (ten ($0.5t$) further).
	\item When a true pixel is found do the same but bottom up .
	\item Remember starting and ending pixel.
	\item If no pixels is found there is no need to start from bottom up.
	\item Bottom up searching cannot go past the starting pixel.
	\end{itemize}
\item Group the vertical lines together.
	\begin{itemize}
	\item Iterate from left to right.
	\item If the left and right one are adjacent and do not differ in length by more than 50\%
	\item If a group is found and large enough remember it and find another group.
	\item Return the two largest groups.
	\end{itemize}
\end{itemize}
\subsection{Motion}
All of the motions designed are keyframe motions, the kicks being a special case. The kick and back kick are given an angle (angle to the goal) as input. The result is an open-loop motion kicking the ball in the specified angle. The keeper movements are of our own design. We use the walk engine by Aldebaran which is a stable omnidirectional walk. Get-up motions are improved versions of the motions provided in Choreographe. 

\subsection{Communication}
\label{sec:communication}
% Sander en Erik, over de gamecontroller (copied Sander's contribution of July27 from GoogleDoc)
At first, the program was tuned to listen to a specific IP address. It would receive the input stream, parse it, and store the information in a GameController object. Recieving the data was done by the \textit{refresh()} method, and the object had several get methods to acquire the data from the object. The refresh method returned true if it had recieved data, and false if it hadn't.

In the Rome Open, the program was made to display the right LED’s on the chest and feet, depending on the current team color and the game state (green for playing, red for penalized, etc.). 

After the Rome Open, the button interface was implemented and combined with the game controller to make a state controller. This is a threaded module which is called to get the game state according to the buttons and game controller (if game controller is present).

Furthermore, the \textit{controlledRefresh} method was implemented, which calls the refresh method repeatedly during a specified interval to see if the game controller is still online (default 4 seconds). If it is, it returns true. If the connection timed out, it returns false, and the state controller starts listening to the buttoninterface, before trying to connect with the game controller again.

In Istanbul it occured, that some teams were broadcasting their own game controllers on the same IP address, which caused problems as the program would listen to all of them. The solution to this was making the program listen to game controllers broadcasting a match with our team number in it only.  

Communication between Naos has been proven to work successfully using built-in methods provided by Aldebaran. It is possible to store a string in memory of a Nao. Other Nao's can receive this specific string if the IP adress is known. This communication will be used to `tell' other Naos how close to the ball they are. Visual information, once processed, can be sent through use of a (not yet determined) protocol, e.g. 'Ball 0.500 0.300 0.583' would mean a ball is at relative position (0.5, 0.3) at distance 0.583 meters.

\subsection{File architecture}
\begin{figure}[htb]    	
\centering
\includegraphics[width=0.7\columnwidth]{cmapFiles.png}
\caption{Hierarchy of files as used in the RoboCup 2011}
\label{fig:File}
\end{figure}
The file hierarchy (see figure \ref{fig:File}) that is used prevents the main program (\texttt{soul.py}) from becoming a chaos. The contents of the files are described in more detail in Appendix~\ref{app:file_overview}.
Motions are specified in a single file. 
For different visual tasks, different files are used. This is mainly to make testing easier (tasks can be tested and programs can be altered simultaneously without trouble due to incompatible versions). 
To make matters even easier, interfaces (\texttt{motionInterface} and \texttt{visionInterface}) are used. These interfaces are used because almost every vision task starts with a snapshot and conversion of this image. 
It would not be aesthetically pleasing to make these function calls in the main program every time a vision call is made. 
The vision interface is well used; the motion interface is implemented merely for completeness (as vision is comparable to motion, both are straightforward tasks that do not alter the flow of the main program directly, unlike the game controller and button interface). The game controller and button interface are each specified in a file to make testing and altering individual functions easier.

\subsection{Localization}
While not yet implemented into \texttt{soul.py} a novel localization method has been developed: Dynamic Tree Localization (DTL). The algorithm is tested with the Gutmann dataset~\cite{Gutmann2002} and is currently extended to work on a NAO.

All the DTL code, the used datasets and the code used to test the algorithm can be found at the website\footnote{\url{http://hesselvandermolen.dyndns.org/BachelorProject/index.php?page=Doc}}. To run the code, Python 2.61, OpenCV 2.1 and Pygame 1.9.1 have to be installed. The code is only tested with the 32bit Windows versions of the necessary programs. It is however assumed that it should be able to work on multiple platforms and with newer versions which are backwards compatible.

\subsubsection{The Algorithm}
Dynamic Tree Localization is a localization method which uses a k-d tree \cite{Bentley1975} to represent the belief of a robot. The root node of the belief-tree represents the complete environment. Each sub-node (child) represents a smaller area of the environment than its parent node. The total surface-size described by all children of a node equals the surface-size of the parent-node:
$$
surface(child_1) \oplus surface(child_2) \oplus ... \oplus surface(child_n) = surface(parent)
$$
To each node a probability is assigned which describes the likelihood that the robot can be found in the represented area. From this, the pose estimation of a robot is calculated. Collapsing and expanding of nodes is done using sets of rules (constraints). For more details about DTL see Appendix~\ref{app:open_challenge_july} and \cite{BscThesisHesselDTL}.

\subsubsection{The Experiments}
To test DTL, the Gutmann dataset is used~\cite{Gutmann2002}. This dataset is a 58 minute recording of a walking AIBO ERS 2100 (equipped with a CMOS camera). The robot walked multiple times an 8-figure in a 3x2m environment. In this environment 6 different color beacons were placed for recognition. The robot was joy-sticked around 5 different mark-points. Each time the robot passed a mark-point, a tag was added to the dataset. Besides the distance and angle towards the color-beacons, the odometry estimation of the robot also got recorded. An overview of the environment is shown in Fig. \ref{fig:GutmannEnvironment}.

\begin{figure}
\centering
\includegraphics[width=9cm]{fieldsetup.png}
\caption{The environment used to create the Gutmann dataset.}
\label{fig:GutmannEnvironment}
\end{figure}

To repeat the performed experiments to determine the average error towards a mark-point for DTL (as described in~\cite{BscThesisHesselDTL}), the file \textit{testLocalization.py} has to be used. By calling \texttt{testLocalization.mainBase()}, the error toward each mark-point for different tree-sizes ($d_{max} \in \{6, 8, 10, 12\}$), collapse (0.1 to 0.6 is steps of 0.1) and expand (0.4 to 0.8 in steps of 0.05) thresholds are calculated. The results are stored in log-files which can be processed by using the \textit{accuracy.sh} script provided by Gutmann\footnote{see: \url{http://cres.usc.edu/radishrepository/view-one.php?name=comparison_of_self-localization_methods_continued}}. Executing the kidnap-test works in a similar way: calling \texttt{testLocalization.mainKidnap(treeDepth, poseFile)} results in a log-file \texttt{poseFile} which contains all information to be able to be processed by Gutmann's \textit{recoverTime.cpp}. The variable \texttt{treeDepth} is the maximum depth of a tree of which the belief consists. In the code the expand and collapse threshold are set to respectively 0.45 and 0.2.

It turned out that a tree with a depth of 8 nodes results in the best performance \cite{BscThesisHesselDTL}. In Fig.~\ref{fig:DTLdepth8} the localization error results are shown. The smallest absolute error is 194mm with expand and collapse conditions of respectively $<0.45$ and $<0.2$. With these conditions a tree with a maximum depth of 10 nodes performs slightly better in accuracy (an error of 193mm) but is slower with re-localization (kidnapping): 4.1 seconds for $d_{max} = 8$ and 4.8 seconds for $d_{max} = 10$. 

\begin{figure}
\centering
\includegraphics[width=8cm]{threshold_t8.png}
\caption{Accuracy-test results with a maximum tree-depth of 8 nodes.}
\label{fig:DTLdepth8}
\end{figure}

Comparing the performance of the current implementation of DTL with the tests performed by Gutmann,~\cite{Gutmann2002}, it can be seen that the absolute error of DTL is twice as big as the absolute error of the adaptive MCL implementation of Gutmann (193mm versus 87mm). The re-localization time is somewhere in the middle of the performance of the by Gutmann tested methods. 

\subsubsection{The Visualizer}
To get a visual overview of the belief of a robot created by DTL, a GUI is build using Pygame: \textit{visualizer.py}. In Fig. \ref{fig:DTLGUI} a screen-shot of the GUI is shown. In the pane on the right side of the GUI the color and name of all landmarks are shown. Left to this pane the color of each probability is shown, from 1 (turquoise) to 0 (dark blue). The left pane shows the environment of the robot, with the landmarks represented as colored circles. The colored rectangles represent a leaf in the robots belief-tree. The color shows the probability and the small circle with line inside a rectangle represents the assumed heading of the robot in the covered area. The large black circle and line represents the ground truth of the robot, while the white circle and line represents the estimated position and heading.

To start the visualizer, execute \texttt{visualizer.initGUI()}. The GUI will immediately start reading observations from the dataset. The GUI consists of two parts. The first part initializes the window, while the second part starts the loop (needed by Pygame) which updates the screen with the new belief.

Adjusting the maximum tree-depth, expand and collapse thresholds is done in \textit{localization.py}. Setting the data-files (for reading and writing) is done in the Pygame-loop in \textit{visualizer.py}.

\begin{figure}
\centering
\includegraphics[width=12cm]{DTL_GUI.png}
\caption{A screen-shot of the GUI for DTL.}
\label{fig:DTLGUI}
\end{figure}


\subsubsection{DTL on a Nao}
Since DTL is build using the same program versions which run on our Naos, DTL itself works on the robot. It however needed a module which returns observations and a server-client connection to show the robots belief, remote, on a computer. For the first a module is build which returns the distance and angle towards color beacons (as used in the Gutmann experiments) given a picture of the Nao's bottom camera. The latter evolved into a server-client structure which is able to stream images from the Nao to a computer.

\paragraph{Beacon Observation /  Landmark detection} \indent \\
To test DTL on a Nao, the Nao needs to be able to detect some landmarks. For this job, six colored beacons are used: pink-yellow-white, yellow-pink-white, blue-yellow-white, yellow-blue-white, pink-blue-white and blue-pink-white. Detection of the beacons is done by combining the blob-detection algorithm, \cite{vanBellen2011}, created for the RoboCup Open Challenge, with functions from the Vision-Module of \texttt{soul}. 

The module (\textit{beaconObs.py}) starts by taking a picture with the Nao's bottom camera. Then, the system applies for each color (blue, pink and yellow) the blob-detection algorithm. The found blobs are combined with each other and if two different colored blobs are within a specified threshold of each other, a landmark is detected. Using the determined pixel-position of the boundary between two colors of a landmark, and a reading of the Nao's pose, the distance and angle towards the landmark in the real world is calculated. In Fig. \ref{fig:BeaconObs} a shot of the landmark detection is shown. The red squares are detected blobs, the blue ones are the detected boundaries between two colors of a landmark (and thus defined as \textit{being} a landmark). As can be seen, a landmark is only detected if the two colored-blobs are above each other (center of the Figure: the horizontal landmark). When two landmark colors of different objects are above each other, the algorithm claims to see a landmark (top right corner: a goal is matched with a landmark).

Calling \texttt{beaconObs.getBeaconObsImg()} activates the landmark detection module. It performs all steps: from taking a picture to calculating the heading and angle and returning the information. The variable \texttt{obs} in \texttt{beaconObs.getBeaconObsImg()} holds the list with distance, angle and landmark signature of the detected landmarks. An overlay of the detected blobs (as shown in  Fig. \ref{fig:BeaconObs}) on the taken picture is stored in the variable \texttt{im} in the same function.

\begin{figure}
\centering
\includegraphics[width=6cm]{beaconDetection.png}
\caption{Beacon detection.}
\label{fig:BeaconObs}
\end{figure}

\paragraph{Adjusting the field and landmark signatures} \indent \\
Adjusting the field size and the landmark signatures with its corresponding location is done in \textit{field\_tree.py}. When adjusting the field size, the dictionary \texttt{D2C} has to be adjusted. If landmarks signatures or landmark positions have to be changed, the variable \texttt{FIELD} needs to be adaptsed. Instruction on how to change these variables can be found in \textit{field\_tree.py}.

\paragraph{Showing remote the Nao's belief} \indent \\
To visualize the robot's belief remote, a client-server connection has to be made. For this connection the \texttt{socket}-module of Python is used. The current communication is not yet fully compatible with the DTL software due to some optimization and changes in the client-server communication. 

The client-server code consists of two different programs: \textit{clientStream.py} and \textit{serverStream.py}. In both files a step-by-step guide can be found on how to setup a connection. The communication between the server and client takes place in turn-take manner: the server sends a request for data (and waits), the client responds by sending the data (and waits for a new command), at which the server in its turn responds with a new request (and waits). In this system the client and the server are only processing information when they receive a command (client) or data (server). After sending information the system goes idle, until it receives again.

Our current implementation is only able to handle OpenCV images. This, however, creates the possibility to see in a real-time manner the result of e.g. filters on a camera image or an image of the belief of the position of the robot.


\subsection{Simulation}

At the Universiteit van Amsterdam a physical realistic model of an Aldebaran Nao robot inside the simulation environment USARsim \cite{Carpin2007icra} is developed. A major advantage of this simulator is that uses the local installation of NaoQiSDK to process commands to the Nao robot, which means that the same python code can be used for the real robot and the simulated robot. 

It was even considered to use a NaoQi system image inside a virtual machine to circumvent the usage of a local NaoQiSDK, but private discussions with Aldebaran employees\footnote{Manuel Rolland and Cedric Gestes} on the RoboCup revealed that Aldeberan itself is busy with this development for their open source release in Fall 2011. 

USARsim is a simulation environment based on the Unreal Engine, which enables easily modification of the environment around the robot (in contrast with Cogmation's NaoSim) and create realistic lighting condition.  

In the previous version of USARsim (based on UT2004) a wide variety of robot models was available, including a Sony Qrio and Aibo. The model of the Nao robot is the first type of walking robot in the new version (based on Unreal Development Kit).  

First a Nao model with two joints (T2) was developed, which was gradually upgraded to a model with 21 joints (T21). The different parts are now nicely scaled, as illustrated in Fig.~\ref{fig:nao_scaled}.

\begin{figure}[hbt]        
\centering\includegraphics[width=.5\columnwidth]{nao33_correctheight.png}
\caption{A model of an Aldebaran Nao robot in USARsim, including a scaled drawing from the Aldebaran documentation}\label{fig:nao_scaled}
\end{figure}

Important for a physical realistic simulation is the calibration of the joints' dynamic behaviors. The dynamic behaviors are configured with parameters such as the mass distribution, the inertia and the gravity. 

Several tests related with gravity, forces and constraints have been performed to validate basic assumptions about the physics of the simulated world. A nice feature of the UDK-model is that the double constraint for the HipYawPitch joint is seemlessly incorporated into the model.

With the center of mass close to the hip, representing a mass distribution according to Aldebaran's specifications, the robot currently can stand up and sit-down. Even more complex behaviors as the Tai Chi and Thriller dance can be executed without loosing balance.

Currently the walking behavior is fine tuned by letting the real and simulated robot walk an eight shaped figure. In the near future the complexity of the tests will be increased, including more complex movements of the Dutch Nao Team.

The development of this open source simulation of a Nao robot not only valuable inside the Standard Platform League, but should also be interesting for the Soccer Simulation League and the @ Home League. Outside the RoboCup community this simulation could be valuable for Human-Robot Interaction research.


\section{Installation Guide}
\subsection{Requirements}
\subsubsection{Program}
The following items are required to run our program \texttt{soul.py}:
\begin{itemize}
\item Nao
\item NaoQi 1.10.52
\item Computer (Windows/Linux)
\item Secure Shell Client (Windows) / SSH (Linux)
\item TurtoiseSVN (Windows) / SVN (Linux)
\item Link to repository\footnote{\url{http://code.google.com/p/dnt-rules/}}
\end{itemize}

\subsubsection{Simulation}
The simulation environment consist of two parts: UsarSim which represents the SoccerField (or another world) where one or multiple robots can be spawned and UsarNaoQi which represents the robot (and translates NaoQi commands into usarsim commands).
For UsarSim an installation guide is available\footnote{\url{http://sourceforge.net/apps/mediawiki/usarsim/index.php?title=Installation}}. For UsarNaoQi an usage guide is available\footnote{\url{http://nao.sandern.com/doku.php?id=usage}}.

The UDK version of UsarSim can be downloaded from sourceforge with the command \texttt{git clone git://usarsim.git.sourceforge.net/gitroot/usarsim/usarsim}, for instance from a Cygwin shell. To allow the usage of the two cameras of the Nao a tool is needed, which is (not yet) available
under git. An update of this tool which uses the video hook (EasyHook) native to UDK to get access of the camera images is available for download\footnote{\url{http://nao.sandern.com/downloads/ImageServer_Beta3.zip}}. This tool should be installed in the directory USARTools.

UsarNaoQi is available for download at \url{http://svn.sandern.com/nao/trunk/usarnaoqi}. The code requires an installation of Visual Studio 2008 and an installation of NaoQiSDK. The location of the NaoQiSDK installation should be specified in the environment variable \texttt{AL\_DIR}.

The latest version of UsarNaoQi is tested with NaoQiSDK 1.10.52.

\subsection{Guide}
\subsubsection{Program}
Follow these steps to install and run the program \texttt{soul.py}:
\begin{enumerate}
\item Download the program from our repository with a SVN program.
\item Move downloaded files from the previous location to the home folder of the Nao (Linux environment).
\item Navigate to folder `home\textbackslash naoinfo' (Windows) or `/home/naoinfo/' (Linux). 
\item Create a file `naoinfo.txt', line 1 is the playernumber, line 2 the teamnumber. Line 1 should be 1 if Nao is a keeper.
\item Navigate to `home\textbackslash naoqi\textbackslash lib' (Windows) or `/home/naoqi/lib' (Linux)
\item Call `python loader.py' from command line.
\item Nao will say that you have to press a footbumper to initiate football mode. Press the bumper.
\end{enumerate}
In this release, there is no safe way to stop the program. Rebooting the Nao or forcing a stop in command line both work but are not pretty solutions to the problem.

\subsubsection{Simulation}
To start the UsarSim including support for the cameras the following command should be given from the UDK directory: \texttt{Binaries\textbackslash Win64\textbackslash udk.exe RoboSoccer?game=USARBotAPI.ImageServerGame -log -d3d11"}.

\begin{figure}[htb]        
\centering\includegraphics[width=.95\columnwidth]{NaoWithTwoCamerasInSoccerWorld.png}
\caption{The  Nao robot in the RoboSoccer environment of USARsim, including the view of both cameras in the upper left corner.}\label{fig:nao_robosoccer}
\end{figure}

To spawn a Nao in this RoboSoccer environment the command \texttt{usarnaoqi.exe} should be given. This program loads its default configuration from the file defaultconfig.xml.

The robot can now be controlled and monitored by any control program, including Aldebaran's Choregraphe and TelePathe (see Fig.~\ref{fig:nao_robosoccer}). The Nao is visible with the default name 'UsarSim1' in the connection windows of both Aldebaran programs.

The simulated robot can also be controlled with the Dutch Nao Team software by typing \texttt{python loader.py} on the commandline.

\section{Sponsors}
The Dutch Nao Team consists of students and most of them do not have enough money to pay for the travel to the Open competitions or the World RoboCup. Therefore, it is essential to get some sponsors. There are many companies that are eager to get some brand awareness and they are prepared to pay for it. It costs time and energy to find the right people from the right companies, but if it is done well, it might lead to a fine collaboration.

\subsection{Preparations}
There are a few things that the team needs to make some decisions on, before somebody starts contacting companies. Firstly, its necessary to know what the team has to offer; banner on the team's website, logo on the t-shirts or on the Nao's etc... Secondly, when the list is ready with all the possibilities that a company may use, prices for the individual options have to be determined. Sometimes people may ask whether there are packages (a few options packed together for a better price) that can be bought. Therefore, it is recommended to make some small packages (consisting of two or three options) for an attractive price, some normal packages and one large package, which contains everything that is offered by the team. This is due to the fact that every company has a different amount of money that they can spend on sponsoring. A larger company may be willing to pay more in contrary with smaller companies, that will try to get a better deal. After making the packages it is recommended to prepare some contracts. This is to prevent that the company has to wait.  It is up to the person who takes care of the sponsoring to know whether to go along with the negotiations or not. When the company is very small, they will not be able to pay as much money as the other companies, but every amount can be helpful fot the team. So it is good to consider your decision about the sponsor.

Another task is to find the companies that might be interested in sponsoring the team. The first choices should be Information Technology (IT) companies, because those are the companies that are primiraly looking for IT students. The internet is the best place to look for them and also to find another businesses that might want to collaborate with the team.

\subsection{Contacting companies}
When all the preparations are finished one can start contacting the companies that have been found. It is highly recommended that every company communicates with one person. This in order to prevent double deals. It also looks unprofessional if more than one person try to contact one person in a company. When a company is interested one should negotiate about what the company wants exactly. Then the contract can be set up and signed.

\section{Public Relations}
\subsection{Getting started}
The Dutch Nao Team is one year old now, which means the Public Relations (PR) department started from scratch, with no script, site or reference to a tournament the Dutch Nao Team had participated in. First of all, the team started with the most general concern of PR: launching the website. Given a website should appeal to a wide variety of people so it should not only describe technical terms, but also introduce the RoboCup and ourselves on a basic level. In addition, the site had to be a clear and easy to use, so everyone can find what they are looking for. To achieve this, a simple menu was made based on the subjects one might wonder about. Furthermore, the language of the site was English instead of Dutch, because of the international character of the RoboCup competition.

Secondly, the communication advisor of the UvA, drs. M. van de Laar, was contacted, who came up with a general script voor PR, which is described in the following section.

\subsection{Making the script}
\subsubsection{The script in general}
Public Relations is all about improving reputation and image. To achieve this, you will need a script revolving the following questions:
\begin{itemize}
\item Audience: Who does the team want to reach?
\item Objective: What does the team want to achieve?
\item Message: What does the team want to communicate?
\end{itemize}

\subsubsection{Audience}
The RoboCup is not very well known outside the robotics community, therefore, choosing a target audience was difficult. The goal was to reach (prospective) students in the teams field to show the possibilities of robotics and perhaps even to recruit new members. Another goal was to promote the RoboCup in general to a bigger audience. The team wanted to address the latter because it would help reaching prospective students and perhaps even sponsors. It was also important to keep the robotics section of the UvA up to date, because without them there would not even be the RoboLab.

\subsubsection{Objective}
In the Netherlands only a few people are familiar with the competition. The objective is thus to promote the reputation of the DNT and promoting the RoboCup. However, it is neither the task nor the responsibility of the DNT to provide a campaign for the RoboCup in general. Nonetheless, the ignorance is something to keep in mind when promoting the team.

\subsubsection{Message}
In short, the message for every group, comes down to:
\begin{itemize}
\item (Prospective) Students: See what the Dutch Nao Team does in their field and join!
\item Public in general: Learn about the possibilities of robotics!
\item UvA: Follow the progress of the team!
\end{itemize}

\subsection{Achievements}
For the past year we have mainly been working on the website and a general view of PR. The biggest obstacle was to decide where to start. Trying to reach our entire audience at once was too much to handle and on top of that we needed different approaches for each one.
\begin{itemize}
\item For our junior prospective students we gave a demonstration at the RoboCup Junior in science center NEMO. This was done to show youngsters the possibilities of robotics.
\item During the RoboCup tournaments everybody wore printed t-shirts to promote the team and improve our reputation within the league. 
\item At the Science Park there were posters of the Dutch Nao Team with information about us, our demonstrations, our progress and of course a reference to our website!
\item To keep contact with our supporters a Facebook-page and a Twitter-account were made. Through these, people from all over the world can follow our progress, thoughts and experiences.
\item And finally, to address the bigger audience, we made a press release about the participation of the DNT in the World RoboCup in Istanbul. We chose this event for the press release because a world cup generally is newsworthy for its international character. This particular press release can be found in Appendix~\ref{app:press_release}.
\end{itemize}

\subsection{Points of improvement for next year}
Next year it would be very wise to start with expanding the script to a much more detailed version. Last year we fiddled about for too long because our ideas were not specific enough and also: sending general messages to an unspecific audience is not efficient or functional at all. 

In addition, the goal to promote the reputation of the RoboCup in general is a job on its own, so it is something to bear in mind when you are writing to people outside the robotics community, but it is definitely not up to us to manage their PR!

On top of the press releases and emails, it is time to make a new, more professional website now that we have finally accomplished something and want it to last.

Furthermore, it is a new years resolution to attend more events for speeches as well as demonstrations.
And last, but not least, we would like to refer to the following site: \url{http://www.freepublicity.nu/}. Unfortunately we ran across this website at the end of this past year, but you can find lots of tips and trics there for the PR department.

\section{Travel}
The Dutch Nao Team travelled to Italy, Iran and Turkey to participate in competitions. For every journey, several arrangements had been made. This section describes, which arrangements were made.

%\subsection{Rome}
% The people participated did not want to share their travel experience and preparations

\subsection{Teheran}
This was a special journey as the Dutch Nao Team got invited by the organizing party (Qazvin University) and the costs for travel and stay were fully covered by them. Therefore, only the flight had to be booked and a visa had to be requested. It is important to do these two on time as both require some time. For the visa, the required documents have to be sent to the organizing committee, which will respond to you when the visa can be requested at the consulate in Den Haag. The organizing committee will make an official invitation letter for the participating members. To request the visa in Den Haag all participating members have to be present at the consulate. Requesting a visa can take up to one day. The flight was booked after a confirmation of the organizing committee was received. Before leaving, a document~\cite{Verschoor2011Iran} of all important facts about the country was sent to all participating members. This was to make sure that participating members were informed about the background of the country, possible required vaccinations, laws and uses of the inhabitants and advice about staying in the country. In Iran, the stay, transport, food, tours of the city were all organized by the committee. 

\subsection{Istanbul}
This journey was to the RoboCup2011 where all international teams participated in a robot football World Cup tournament. The flight and hotel were booked quite late causing the price of the flight to be higher. A nice hostel was found nearby the location of the event (name of the hostel is Han hostel, located near Atatürk airport and the Expo Center, where the RoboCup was held). Again, a document~\cite{Verschoor2011Istanbul} was made with important facts about the country. Next to that a packing list with necessary materials was made. In Istanbul, breakfast and lunch was bought from the nearby supermarket. Dinner was bought and consumed at nearby restaurants. The shuttle service of the hostel was used to travel from hostel to Expo Center and vice versa. Public transport and the taxi were used only to travel into the city for sightseeing. The last day of the journey was spend sightseeing around the city of Istanbul. Several cultural locations were visited by most of the participating members.

\subsection{Tips for the next journey}
This section describes the advice to the organizers of next year. The following points are important to take into account for next year:
\begin{itemize}
\item Have strict deadlines for participating students. Pay and sign up deadlines are both important as without money you cannot book the flight and hotel.
\item Start on time with booking the flight to reduce costs.
\item Find a hostel/hotel near the location of the event.
\item Write a travel information document. So all the important information bundled together.
\item Communicate clearly with the students participating in the team.
\end{itemize}

\section{Management}

\subsection{Season 2010-2011}
At the start of the project, the Dutch Nao Team (DNT) was split into different groups: {\it Vision}, {\it Motion}, {\it Communication} and {\it Architecture}. Each group consisted of a captain and sufficient members to complete their work. The captains were in control of the groups, as well as being a programmer. Above the captains were two other positions, the manager and advisor. Since the team was large at first, splitting the team seemed like a good idea at the time. It did work because the team started from scratch and had many work to do. After some time, this did not go so well. It became clear that the groups were dividing the DNT by blaming each other for bugs, loss of files and other complications. It also seemed that each group saw itself as a standalone group instead of cooperating with the rest of the team.
Halfway through the year, this setup was changed. Pairs were formed, which would then work on a single task (e.g. goal recognition, motion). These pairs worked independently and would add their programs only when finished. We also switched from Dropbox to SVN, a huge improvement which made the management of files easier (and tidier as well). 

\subsection{Season 2011-2012}
The major problem with participation last year was that it was voluntary, this made it hard to demand progress when someone was behind on his work. The second year there will only be members in the team that contribute actively. If anyone outside the team wants to help or contribute to the RoboCup through related work they still can but that will be either on minor projects that do not have a deadline or by helping existing pairs. Members that are unable to do the minimal required work for an extended period of time will have to talk with the manager and advisor, who will decide if a person can stay or has to leave the Dutch Nao Team..

A second problem was that the team started as a large group (23 people). As everyone, at that point, lacked experience in managing, the focus lies on keeping the team small. This makes management and organization easier and everyone in the team will feel more involved.  

\subsection{RoboCup Junior/Dutch Nao Demo Day}
On May 28th, the DNT had a small stand at the RoboCup Junior at NEMO, where the most recent code was showed off, by letting Naos play soccer on a small pitch which was open for public throughout the day. In collaboration with Dutch Nao Day a small demonstration was given.
 
The goal of the Dutch Nao Demo Day is to bring different Nao-owners that live in the Netherlands together, show each other the demonstrations created in the different institutes in the Netherlands and create - in a shared effort - one big demonstration, which is shown at the end of the day in the theatre of the science center NEMO. This was and (will hopefully remain) an excellent opportunity to show the efforts of the Dutch Nao Team to the public. 

\section{Evaluation and Future Improvements}
The Dutch Nao Teams debut in the RoboCup2011 exceeded our expectations. Not only did we score the very first official goal (6 goals in total), we also improved almost every aspect of the code during and after the RoboCup. We found that playing a match in real life is the best evaluation you can have. 

\subsection{Future work}
The Dutch Nao Team has only worked with open-loop motions until now. The main goal for 2012 is to either create a stable closed-loop walking motion (or base it on Aldebarans omnidirectional walk) or create a fully dynamic kick. Falling during a kick was one of the main problems in this years RoboCup. A faster gait would be an important improvement because this teams robots are currently the slowest in the competition.

Communication between players, which was not used during this RoboCup, is already possible through modules provided by Aldebaran. It is quite useful - even without localization. During the matches, all players (except the goalkeeper) would all move towards the ball simultaneously, causing them to collide, fall or be penalized for pushing other players. Deciding which player should move towards the ball is possible if information about the relative location of the ball is shared with other players.

Nao-recogntion was already implemented for use in the ZoneBlocking project (see section \ref{sec:zoneblocking}. However, the algorithm as it is now is a slow, straightforward process which can not be used if it is not improved or rewritten. Ball-recognition and the new goal-recognition algorithm (see section \ref{sec:vision}) could also be made more efficient by giving motion calls for the neck joint are during calculations instead of afterwards.

\subsubsection{Line Detection}
To find a line on the field, an image is filtered on white pixels (lines are white). In this filtered image, the Houghlines algorithm, found in OpenCV \cite[p.~156]{Bradski2008}, can find lines. Using (intensity) thresholds, lines with corresponding accumulator value greater than the threshold are returned. It is also possible to input a minimal required length for a line. When a line is partially invisible, the algorithm can reconstruct the line. An additional parameter is the maximum length a gap between lines can be before its not considered to be the same line. 
Unfortunately, the Houghlines algorithm returns a lot of redundant lines. This is caused by the different thicknesses of lines in a 3D world projected onto a 2D plane. A different threshold for lines further away than lines close by is needed, as currently, the same line close by looks thinner when its further away. The problem that that causes is that the algorithm finds multiple lines on a single line simply because it is so thick that it is not considered to be the same one. To reconstruct these thicker lines, lines that are close to each other, with the same angle, are appended to form a single line. 
Corners of the field and T-junctions can be found using the data the Houghlines algorithm returns. The Nao can use these as landmarks for localization. The angles found for T-junctions will also contribute to the calibration of the camera, because the 2 lines should always be perpendicular in the real world. The angle between the two lines that form a T-junction found through this algorithm will not always be 90 degrees. Therefore, this angle can be used to recalibrate the camera in order to maximize accuracy of localization. 

\subsubsection{Kinect your Nao: A human interface}
In June 2011, a second-year project involving the Naos, an XBox Kinect and virtual reality glasses was started. This project had as goal to let a Nao mimic a human operator. Complications involved the balance issues when mimicking legs, the different platforms that were used (as python was used to control the Nao and C++ to extract data from other devices). While largely avoiding the balance issue until closed-loop motions are started, the problem of multiple programming languages was solved through use of a TCP connection between Nao and corresponding programs. 

The TCP server that runs on the Nao is the core of the system. It can connect to a large number of clients which can all send data as long as the protocol for messages is followed. The messages should always be strings so there is no limit to the possible programming languages, as long as a connection to the server can be made. Duplicate messages are not a problem either, since the program only executes the first command in a message containing duplicate (and thus possibly contradicting) joints and angles. The only constraint is that the buffer for the messages is not infinitely big, a message from a client can, in the current program, not be larger than 4096 bytes. The slowest client determines how fast the Nao is able to execute given commands, since the server waits for one message from every client before updating the commands.

After initiating the program with the number of connections as input for the start-function, clients can connect to the TCP server. Once the inputted number of connections is reached, the Nao will start executing commands from the clients.
The first client that was implemented is a program collecting data from the Xbox Kinect. It calculates the angles for the joints of the Nao that correspond to the angles of the users body. This is done through positions of joints involved in that angle. From these positions vectors are calculated. Using the dot product of the two vectors, the angle between the angle between the two is determined.
To illustrate this approach: If one wants to calculate the roll of the right shoulder, take the positions of the right elbow($p_{E}$), the right hip($p_{H}$) and the right shoulder($p_{S}$) in order to calculate $\angle p_{E}p_{S}p_{H}$.

Normal trigonometry is used to solve this, because all the points in space are known and thus, the angles are easily calculated. Instead, the dot product of vectors is used. This is because for some joints, three points describing the joint cannot create a triangle in a single plane. For example, the elbow yaw cannot be calculated in a triangular fashion. The elbow yaw is the joint that makes you able to turn you lower arm around the axis of your upper arm. So if it is necessary to create a triangle that calculates the elbow yaw, the positions of the elbow, the hand and another joint in that plane that is perpendicular to the upper arms axis are needed. There is no such joint, therefore the angle can’t be calculated through trigonometry. A solution to this problem is that vectors are used, which do not have to have a common point in order to calculate the angle between them. For instance, for the elbow roll, the vector pointing from the shoulder to the hip and the vector pointing from the elbow to the hand are used. If the dot product between the two is calculated, the angle can easily be determined (where $\theta$ is the angle between the two vectors):
$$ \vec{a} \cdot \vec{b} = ||\vec{a}|| ||\vec{b}|| \cos{\theta}$$
The second client collects data from the gyroscope in the Vuzix virtual reality glasses. This data is used to determine two angles, a yaw and pitch, for the head of the Nao (the Nao does not have a neck joint for the roll). It also acts as visual input for the user, acting as a second screen for the computer it connects to.
To create this visual input, Choreographe, a program by Aldebaran robotics, is used. The benefit of this program is that it can simultaneously show the camera images and the pose of the Nao in simulation. The latter because the real Nao’s body is not visible to users with the glasses on.
A third client collects gyroscope data from two Wii remotes, as well as information about button presses. This data is used to control the Nao’s hands and wrists (only for Nao Academic version, which has motors in wrists and hands) and to make the Nao walk. Though the use of buttons does not contribute to the idea of a human interface, it is a useful addition to the possibilities of control of the human operator.

With the components that are currently used (the Xbox Kinect, the Vuzix virtual reality glasses, the Wii remotes, the Aldebaran Nao and Choreograph), the program gives the user a camera view and the pose of the Nao in simulation through the glasses. The Nao itself will mimic the users upper body which is made possible through use of the Kinect and Wiimotes, though this can only be seen in simulation if the user is wearing the VR glasses. Gestures or buttons on the Wii remotes activate walking of the Nao. This enables the user to execute simple remote operations, such as lifting a box or pushing a button. It should be noted that calibration of the VR glasses is needed before use of the system with all it’s current components.

\begin{figure}[htb]
\centering
\fbox{\includegraphics[width=1\columnwidth]{KiNao_Poses.jpg}}
\caption{The system in use}
\label{fig:Duncan}
\end{figure}

The system implemented allows a human user to control humanoid robots through body motions (see figure \ref{fig:Duncan}). It does not require the user to lift the legs or walk, only the upper body is mimicked. Requirements for use are perhaps a few hours of practice but no knowledge of robotics is needed. It is therefore applicable in any area. A few examples of possible use are teleoperation in dangerous area’s (e.g.\ after natural disasters or warzones), house-holding tasks but also industrial processes. The human interface makes it relatively easy to learn the robot a new task since it can simply mimic the user executing the task. This task is relevant to the RoboCup precisely because of that. While not currently possible, balancing while mimicking a human operator drastically shortens the time needed to design a motion. A user can simply execute a motion while seeing the Nao mimic it. This motion is then recorded and smoothed (removing noise due to inaccuracies by the Kinect, removing control points, etc.). This simplification of motion design makes it possible for an inexperienced (meaning no experience with Naos) user to create basic, but smooth, keyframe motions.

\section{Acknowledgements}
The Dutch Nao Team likes to thank the following people:
\begin{itemize}
\item Sander van Noort for his contribution to the simulation. 
\item The University of Amsterdam for their support, the new lab, the contributions and experience we get from this project. 
\item All our Artificial Intelligence teachers for when they took RoboCup into account.
\item Adwin Verschoor of Cena-Werk\footnote{Visit him at \url{http://www.cenawerk.nl} for computer support} providing us with full technical support. 
\item The Italian and Iranian National Committee for their hospitality and travel support
\item RoboCup Junior for a great event
\item Marthy van de Laar for her help with Public Relations
\item The Standard Platform League for blessing us with the Robocup.
\end{itemize}
Last but not least we would like to thank: Sander Latour, Patrick de Kok, Chiel Kooijmans, our parents and everybody that has helped for their contributions.
\newpage
\bibliographystyle{mystyle}
\bibliography{references}

\section{Appendix}
\subsection{Team Poster}

\fbox{\includegraphics[height=0.9\textheight]{PosterRoboCup2011.jpg}}

\subsection{Open Challenge Document June}
\label{app:open_challenge_june}

\fbox{\includegraphics[height=0.9\textheight]{RoboCup2011-OpenChallengeDNT.pdf}}

\subsection{Open Challenge Document July}
\label{app:open_challenge_july}

\fbox{\includegraphics[height=0.9\textheight]{OpenChallenge.pdf}}

\subsection{Press release}
\label{app:press_release}
\fbox{\includegraphics[height=0.85\textheight]{DNTpersbericht.pdf}}

\subsection{File Overview}
\label{app:file_overview}

\fbox{\includegraphics[width=\columnwidth]{FileOverviewPage1.png}}

\fbox{\includegraphics[width=\columnwidth]{FileOverviewPage2.png}}
\end{document}
